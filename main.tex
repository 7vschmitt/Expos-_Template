\documentclass[11pt]{article}
\usepackage[T1]{fontenc}
\usepackage[utf8]{inputenc} %choose utf8 encoding
\usepackage[english]{babel}
\usepackage[explicit]{titlesec}
\usepackage[a4paper,width=160mm,top=35mm,bottom=25mm,bindingoffset=6mm]{geometry}
\usepackage[labelfont={footnotesize,bf} , textfont=footnotesize]{caption}
\usepackage{fancyhdr}
\usepackage{graphicx}
\usepackage{csquotes}


\usepackage[style=apa]{biblatex}

\addbibresource{sample.bib}


%%%%%%%%%%%%%%%%%%%%%%%%%%%%%%%%%%%%%%%%%%%%%%%%%
\begin{document}


\input{title}
%please go to title.tex the file to change all the information accordingly


\section{Motivation}

Please describe shortly the main challenges why the topic you are working on is relevant from a scientific perspective. Include some state-of-the-art literature in order to argue for the relevance of the gap in the literature you want to fill. 

The second paragraph should shortly introduce how you will fill the gap in the literature and about your scientific contribution. Here you can introduce your research questions and the exact challenge you want to tackle.

The third paragraph should be about a very brief outlook of the structure of what will be covered in the Exposé. 


\section{Related Work/Literature Review}

In this section you should some up the most relevant work in the domain you want to write your thesis in. Imagine a funnel, where you start with a broad overview and getting more and more concrete until you end up with your specific research question(s). Hereby, it is important, not just to summarize previous findings, but also to interpret them in the context of the challenges you want to work on. 


\section{Goals}

In this section please answer the following questions: 

\begin{itemize}
    \item What goals will be achieved within the thesis project? 
    \item What is the scientific contribution of the thesis project? 
    \item What are the expected results? 
\end{itemize}

\section{Methodology}

Shortly describe the methods you are using within your thesis project to get to the expected results you described previously.

\section{Time Planning/Thesis Structure}

This is one of the core elements. Here you should describe the elements of your thesis. You can visualize them in a graph on a timeline, where it will be clear which elements build on each other and how long you are planning to work on which step. This part will be one of the building blocks where the supervisors should get a clear idea what will be done in the scope of the thesis and when to expect some results of the respective elements/blocks. Also add a structure of your how your thesis will look like in the end.




\textbf{If you add some references in the sample.bib file you can refer to them by the cite command (e.g. \cite{moore2008defining}).}

\printbibliography

\end{document}
